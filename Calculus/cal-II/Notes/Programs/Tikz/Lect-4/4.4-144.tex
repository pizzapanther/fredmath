 
\documentclass{minimal}
\usepackage{tikz}

\usepackage{verbatim}
\usepackage[active,tightpage]{preview}
\usepackage{pgfplots}
\pgfplotsset{compat=1.13}
\usepackage[T1]{fontenc}
\PreviewEnvironment{tikzpicture}
\setlength{\PreviewBorder}{12pt}

\usepackage{mathtools}
\usepgflibrary{fpu}

\begin{comment}
:Title: Polar Exponential

\end{comment}
\usetikzlibrary{calc}

\begin{document}
\begin{tikzpicture}

	%\draw[thick, domain=-3:4] plot({\x},{( exp(-\x/6))}) ;

%	\draw [ line width=1mm, color=blue, domain=-2*pi:2*pi, samples=200, smooth]
%		plot (xy polar cs:angle=\x r, radius={exp(\x r /6) });

	\node [ scale=1.6, color=blue] at (3,3) {$r=e^\theta /6$};
	\fill [fill=red!80!black, opacity=0.5] plot [domain=pi/2 :3*pi/2 ] 
		 (xy polar cs:angle=\x r, radius= {2+2*cos(\x r)});

	 \draw[blue, line width=1mm, domain=-2*pi: 2*pi, smooth, variable=\t, samples=50]  
		plot({cos(\t r+\t)}, { sin(\t r +\t)  });
	
	%\draw [ line width=1mm, color=black!60!green, domain=0:2*pi, samples=200, smooth]
  	%	plot (xy polar cs:angle=\x r, radius={2-2*cos(\x r)});
	%\node [ scale=1.6, color=black!60!green] at (-3,3) {$r=2-2\cos\theta$};	
	%\fill [fill=red!80!black, opacity=0.5] plot [domain=-pi/2 :pi/2 ] 
	%	 (xy polar cs:angle=\x r, radius= {2-2*cos(\x r)});

	\draw[->, ultra thick] (-4.8, 0) -- (4.5, 0);	% x-axis
	\draw[->,  ultra thick] (0, -2.7) -- (0, 3.5);	% y-axis
	% Points
	\draw [scale=1.4] (0.1, 1.7) node {$2$};

\end{tikzpicture}
\end{document}