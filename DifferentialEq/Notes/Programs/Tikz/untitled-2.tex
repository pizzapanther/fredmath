\documentclass{article}

% PSTricks.
\usepackage{
  pst-circ,
  pst-bspline,
  pstricks-add
}
\psset{unit = 0.8}

% Width and height of the diagram.
\def\width{12 }
\def\height{6 }

% Calculating width and height of picture.
\usepackage{expl3}
\ExplSyntaxOn
  \cs_new_eq:NN \calc \fp_eval:n
\ExplSyntaxOff
\newcommand*\Width{\calc{\width+0.4}}
\newcommand*\Height{\calc{\height+0.82}}

\begin{document}

% Drawing.
\begin{pspicture}(-0.4,-0.05)(\Width,\Height)
  % Coordinates for the ends of the wires.
  \pnodes{P}%
    (0,0)(0,0)% Doubled to let the nodes start from P1 instead of P0.
    (!0.7 \width mul 0)%
    (\width,0)%
    (0,\height)%
    (!0.35 \width mul \height)%
    (!0.7  \width mul \height)
    (\width,\height)
  % Wires and components.
  \wire[arrows = o-o](P1)(P3)
  \coil[arrows = o-](P4)(P5){$L$}
  \resistor[arrows = -o, dipolestyle = zigzag](P5)(P6){$R$}
  \wire[arrows = o-o](P6)(P7)
  \capacitor[arrows = *-*](P2)(P6){$C$}
  % Labels.
  \uput[90](P1){$-$}
  \uput[90](P3){$-$}
  \uput[270](P4){$+$}
  \uput[270](P7){$+$}
 {\psset{linestyle = none}
  \pcline(P1)(P4)
  \ncput*{$v_1(t)$}
  \pcline(P3)(P7)
  \ncput*{$v_2(t)$}}
  % Direction of the current.
  \psBspline[arrows = ->]%
       (!0.7 \width mul 3   sub \height 1 sub)%
       (!0.7 \width mul 1.2 sub \height 1 sub)%
       (!0.7 \width mul 1.2 sub 1)%
       (!0.7 \width mul 3   sub 1)
  \rput(!0.7 \width mul 2.2 sub \height 2 div){$i(t)$}
\end{pspicture}

\end{document}